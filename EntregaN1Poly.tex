\documentclass[a4paper,12pt]{article}
\usepackage[utf8]{inputenc}
\usepackage{amsmath}
\usepackage{hyperref}
\usepackage{graphicx}

\title{Entrega N1 - Metodologia de Pesquisa em Computação}
\author{}
\date{06P}

\begin{document}

\maketitle
\tableofcontents
\newpage


\section{Integrantes}
\begin{itemize}
    \item Enrico Minto Spanier - 10419775
    \item Guilherme Vecchi Bonotti Freitas Silveira - 10418517
    \item Alexandre Luppi Severo e Silva - 10419724
\end{itemize}

\section{Tema ou Área Estabelecida para o TCC}
Uso da Estrutura de Microserviços em Apps com Marketplace e Gerenciamento de Eventos, e Uso de API de Integração com Redes Sociais

\section{Delimitação da Área}
O trabalho abordará a aplicação de arquitetura de microserviços no desenvolvimento de aplicativos de marketplace, que incluirá tanto a venda de produtos diretamente relacionados a eventos quanto produtos externos aos eventos. Além disso, o projeto focará no gerenciamento de eventos dentro da plataforma e na integração com redes sociais por meio de APIs. Serão explorados conceitos de escalabilidade, manutenção, segurança, e as vantagens dessa estrutura arquitetural para o desempenho de aplicativos em contextos comerciais digitais, garantindo a flexibilidade necessária para lidar com diferentes tipos de produtos e interações entre usuários e organizadores.

\section{Nome do Futuro Orientador de TCC}
\begin{itemize}
    \item Thiago Graziani Traue
\end{itemize}

\section{Relatório de Levantamento Bibliográfico}

\subsection{Palavras-chave Definidas}
\begin{itemize}
    \item Microserviços em marketplace
    \item Gerenciamento de evento em marketplace
    \item Integração de redes sociais em marketplace
    \item Integração de APIs com marketplace
    \item Microserviços para aplicações de gerenciamento de evento
    \item Arquitetura de microserviços distribuídos
    \item E-commerce com microserviços
    \item Arquitetura escálavel para marketplace
    \item Segurança de API para integração de redes sociais
\end{itemize}

\subsection{Strings de Busca Elaboradas}

\subsubsection{Microserviços em Marketplace}
\begin{itemize}
    \item "Microservices architecture" AND "marketplace application" AND ("scalability" OR "performance")
    \item "Microservices" AND "marketplace" AND "event management" AND "e-commerce"
    \item "Distributed microservices" AND "marketplace" AND "product management" AND "API"
\end{itemize}

\subsubsection{Gerenciamento de Evento em Marketplace}
\begin{itemize}
    \item "Event management" AND "microservices" AND "marketplace" AND ("platform" OR "application")
    \item "Event management system" AND "marketplace" AND "microservices" AND ("integration" OR "scalability")
    \item "Event management" AND "microservices" AND "scalable architecture" AND "e-commerce"
\end{itemize}

\subsubsection{Integração de Redes Sociais em Marketplace}
\begin{itemize}
    \item "Social media API integration" AND "marketplace" AND "microservices" AND ("scalability" OR "security")
    \item "Social media APIs" AND "marketplace" AND "e-commerce" AND "integration" AND "microservices"
    \item "Marketplace" AND "social media" AND "API security" AND "microservices"
\end{itemize}

\subsubsection{Integração de APIs com Marketplace}
\begin{itemize}
    \item "API integration" AND "microservices" AND "marketplace" AND ("security" OR "performance")
    \item "API integration" AND "microservices" AND "marketplace" AND "event management"
    \item "Marketplace application" AND "API integration" AND "microservices" AND ("e-commerce" OR "scalability")
\end{itemize}

\subsubsection{Microserviços para Aplicações de Gerenciamento de Evento}
\begin{itemize}
    \item "Microservices" AND "event management" AND "marketplace" AND "scalable" AND ("API" OR "integration")
    \item "Event management" AND "microservices" AND "distributed architecture" AND "marketplace"
    \item "Event-driven architecture" AND "microservices" AND "marketplace" AND "scalable"
\end{itemize}

\subsubsection{Arquitetura de Microserviços Distribuídos}
\begin{itemize}
    \item "Distributed microservices" AND "marketplace" AND "event management" AND ("scalability" OR "performance")
    \item "Microservices" AND "distributed architecture" AND "marketplace" AND ("scalability" OR "security")
    \item "Distributed microservices" AND "API" AND "marketplace" AND "integration"
\end{itemize}

\subsubsection{E-commerce com Microserviços}
\begin{itemize}
    \item "E-commerce" AND "microservices" AND ("marketplace" OR "event management") AND ("scalability" OR "security")
    \item "E-commerce platform" AND "microservices" AND "API integration" AND ("marketplace" OR "event")
    \item "Microservices" AND "e-commerce" AND "marketplace" AND ("security" OR "performance")
\end{itemize}

\subsubsection{Arquitetura Escalável para Marketplace}
\begin{itemize}
    \item "Scalable architecture" AND "microservices" AND "marketplace" AND "performance"
    \item "Scalability" AND "microservices" AND "marketplace" AND ("event management" OR "e-commerce")
    \item "Scalable architecture" AND "microservices" AND "marketplace" AND "API integration"
\end{itemize}

\subsubsection{Segurança de API para Integração de Redes Sociais}
\begin{itemize}
    \item "API security" AND "social media" AND "integration" AND "microservices" AND "marketplace"
    \item "API security" AND "microservices" AND "social media" AND "e-commerce"
    \item "API security" AND "social media" AND "integration" AND "marketplace" AND "microservices"
\end{itemize}

\subsection{Informações Completas dos 15 artigos pesquisados e selecionados}
Acessar as bases de dados Web of Science (WoS), SCOPUS, ACM e IEEE (por meio do Portal Periódicos), assim como as Bibliotecas Digitais e Físicas, delimitar o escopo dos resultados e selecionar, pelo menos, 15 referências bibliográficas (sendo que cada uma delas deve dispor uma justificativa, de três a cinco linhas, para a pertinência em relação ao estudo).


\end{document}
